\section{Introduction}

\newpage

\subsection{Votre mission}

\enluminure{C}{hevalier, votre mission}, car vous l'avez acceptée~: mener à bien
pour notre Roy une mission de la plus haute importance.

\begin{center}
  {\large\textsc{Conquérir la Terre du Centre.}}
\end{center}

La Terre du Centre, ce pays légendaire et convoité. Depuis des
générations, beaucoup de nobles chevaliers et de braves soldats s'y sont
risqués~; aucun n'en est jamais revenu.

Aujourd'hui, c'est à vous que revient cet immense honneur que d'y
planter la bannière royale.

Soyez grands, soyez forts, soyez courageux et défendez nos valeurs
face à nos adversaires.\\

Et n'oubliez pas Chevalier~:

\begin{center}
  \textsc{L'extinction n'est pas une option.}
\end{center}

\subsection{Le champ de bataille}

Bien que d'origine 100\%{} naturelle\footnote{et élevée en plein air,
  label rouge}, la Terre du Centre se découpe admirablement bien en
cases carrées, se groupant elles-mêmes en zones. Parce que notre bon
Roy est également fin stratège, il vous indiquera à chaque fois quelle
zone vous devrez conquérir et défendre.
\newpage

\section{Les Règles de l'Engagement}

\subsection{Le champ de bataille}
\prolocitation{--- Et ça, c'est la Terre du Centre~?\\
--- Effectivement. C'est la Terre du Centre.\\
--- Elle n'a pas l'air terrible.\\
--- Faut pas s'y fier, car personne n'en est ressorti~!}

Les territoires de la Terre du Centre sont composés de différentes
cases, pour différents terrains. Vous serez ainsi amené à rencontrer :

\begin{itemize}
  \item de l'\textbf{herbe}, terrain de base qui ne procure ni
    avantage ni handicap ;
  \item des \textbf{routes}~: le déplacement sur une route ajoute un
    bonus à la distance maximale de déplacement ;
  \item des \textbf{forêts}~: les forêts sont épaisses dans la Terre du
    Centre, et limitent fortement le champ de vision ;
  \item des \textbf{marais}~: le déplacement dans les marais est
    difficile, et impose un malus à la distance maximale de
    déplacement ;
  \item des \textbf{murs}~: dressés par des entités
    surnaturelles\footnote{selon la légende, elles porteraient des
      tuniques noires}, les murs sont absolument infranchissables (et
    opaques, bien entendu) ;
  \item des \textbf{tours de guet} : une tour de guet apporte des
    bonus en distance de vision et portée de tir.
\end{itemize}

Chaque case peut accueillir un nombre illimité de personnages (amis et ennemis).

\subsection{Vos hommes}
\prolocitation{}

Toutes les unités peuvent attaquer, au maximum une fois par
tour. Elles ont également deux capacités spéciales chacune, décrites
un peu plus loin.

\begin{itemize}

\item \textbf{Le voleur}

\prolocitation{Je ne suis pas un voleur, je suis président de la République}

Le voleur, agile par nature, se déplace rapidement (FIXME nb Pt /
case). Son champ de vision est plus important que les autres unités
(FIXME cases, image).

Points d'Action : //FIXME

Type d'attaque : corps-à-corps (1 case)

\item \textbf{Le barbare}

\prolocitation{--- Tiens donc, un paysan\\
--- Chuis un barbare~!\\
--- Aucune différence~!\\
(NDLR\footnote{Devant la violence de cette scène et la présence de
    mineurs parmi notre public, nous avons fait le choix de ne pas
    diffuser ces images. Merci de votre compréhension.})}

Le barbare se déplace à vitesse normale (FIXME nb Pt / case), et a un
champ de vision standard (FIXME cases, image).

Points d'Action : //FIXME

Type d'attaque : dégâts de zone

\item \textbf{Le tireur d'élite}

\prolocitation{--- Alors, qu'est-ce que tu as gagné ave ton niveau ?\\
--- Je crois que je tire mieux à l'arc.\\
--- Hahaha !\\
--- Ben quoi, c'est drôle ?\\
--- Ouais, c'est assez drôle quand tu dis ça.}

Le tireur d'élite est une unité lente mais efficace. Il peut se
déplacer à hauteur de (FIXME nb Pt / case), et dispose d'un champ de
vision standard (FIXME cases, image).

Points d'Actions : //FIXME

Type d'attaque : distance (FIXME)

\end{itemize}

\subsection{La Mort}
\prolocitation{Tagazok !}

Cruelle destinée, la mort n'en est pas moins indispensable : c'est en
effet grâce aux décès des uns et des autres que les champions gagnent
des points\footnote{Les malheurs des uns font le bonheur des autres\ldots{}}.\\

Posons tout d'abord une base importante : un personnage meurt lorsque
la valeur de ses points de vie devient négative ou nulle.

Trois règles déterminent les points attribués à la mort d'un
personnage. Les voici par ordre de précédence :

\begin{enumerate}
\item Le personnage meurt suite aux attaques d'un adversaire
  \textsc{et} d'un allié : aucun point n'est accordé.
\item Le personnage meurt suite à un dommage collatéral uniquement : un point est
  retiré à son champion.
\item Le personnage meurt suite à un tir adverse uniquement : un point
  est attribué à l'attaquant.
\end{enumerate}

\subsection{Le champ de vision}
\prolocitation{Loin des yeux, loin du c\oe{}ur\ldots{}}

\subsection{Les déplacements}
\prolocitation{--- Descendez l'escalier, et c'est tout droit.\\
--- Mais ça veut dire que\ldots{} Si on avait pris tout droit en
entrant, on serait arrivé ici directement ?\\
--- Bien sûr !}

\subsection{Les capacités spéciales}

\begin{itemize}
\item \textbf{Le voleur}
  \begin{itemize}
  \item \textit{Palantir}

    Description de Palantir

    Délai entre chaque utilisation : 3 tours
  \item \textit{Dans ton dos}

    Description de Dans ton dos

    Délai entre chaque utilisation : 8 tours
  \end{itemize}
\item \textbf{Le barbare}
  \begin{itemize}
  \item \textit{Bastoooon}

    Description de Bastoooon

    Délai entre chaque utilisation : 3 tours
  \item \textit{Fus ro dah!}

    Description de Fus ro dah!

    Délai entre chaque utilisation : 10 tours
  \end{itemize}
\item \textbf{Le tireur d'élite}
  \begin{itemize}
  \item \textit{I see what you did there}

    Description de I see what you did there

    Délai entre chaque utilisation : 5 tours
  \item \textit{Loto, à qui le tour}

    Description de Loto, à qui le tour

    Délai entre chaque utilisation : 8 tours
  \end{itemize}
\end{itemize}

\newpage
\section{Le déroulement d'un match}

Chaque match se décompose en deux phases d'égale importance : le
placement et le jeu.

\subsection[La fin du jeu]{La fin du jeu\protect\footnote{Oui, on sait...}}

Un match dure //FIXME tours. Est déclaré vainqueur le champion ayant
gagné le plus de points.

\subsection{La phase de placement}
La phase de placement n'intervient qu'une seule fois, au début du
match. C'est une phase exclusivement pacifique, où les personnages se
déplacent jusqu'à atteindre un emplacement stratégique\footnote{C'est
  en tout cas tout le mal que l'on vous souhaite !}.
Au démarrage de chaque match, les trois personnages des trois joueurs
sont placés sur une même case (généralement au centre de la carte,
mais pas forcément).

À l'issue des //FIXME tours attribués pour le placement, la position
courante de chacune des unités devient son point de réapparition après
la mort.

Choisissez bien votre chemin, ce sont les derniers instants de
tranquillité dont vous disposez !

\subsection{La phase de jeu}
Chaque tour de jeu est divisé en deux étapes, le jour et la nuit, où
vous pourrez donner différents types d'ordres à vos
personnages. Néanmoins, les ordres des deux phases sont donnés au
début de chaque tour. L'état intermédiaire entre les deux reste inconnu.

\subsubsection{Étape 1 : attaques}
\prolocitation{Aux armes, citoyens\\
Formez vos bataillons}

Les attaques ont lieu en plein jour. Tout ce que les personnages ont
pu apercevoir lors de la nuit précédente devient disponible à ce moment-là.

Les ordres d'attaque doivent donc tenir compte des déplacements de la
nuit précédente.\\

Les attaques sont portées contre des cases : il est donc possible de
blesser un allié se trouvant au même endroit, voir même de s'infliger
des dégâts. La case ciblée n'est pas identifiée par ses coordonnées
absolues, mais par ses coordonnées relatives à la position de l'attaquant.\\

Toutes les attaques sont lancées en même temps, à l'exception de Fus
ro dah!, qui est prioritaire.

\subsubsection{Étape 2 : déplacements}
\prolocitation{Marchons, marchons !}

La nuit est en revanche propice aux déplacements d'unités. Protégés par
l'obscurité, l'allégance et le type des unités rencontrées en chemin
resteront secrets. Le lendemain, vous n'aurez accès qu'au chemin
emprunté par vos rencontres nocturnes.

Les personnages se déplacent en même temps. Les plus rapides
attendront donc les plus lents\footnote{Les plus casaniers attendront
  donc les grands voyageurs, ça marche aussi.}.


\newpage
\section{Conclusion}

\prolocitation{Voix off --- Vous pensez peut-être que c'est terminé ?\\
Ranger --- Bah ouais.\\
--- Et bien oui.}
