\section{Introduction}

\newpage

\subsection{Votre mission}

\enluminure{C}{hevalier, votre mission}, car vous l'avez acceptée~: mener à bien
pour notre Roy une mission de la plus haute importance.

\begin{center}
  {\large\textsc{Conquérir la Terre du Centre.}}
\end{center}

La Terre du Centre, ce pays légendaire et convoité. Depuis des
générations, beaucoup de nobles chevaliers et de braves soldats s'y sont
risqués~; aucun n'en est jamais revenu.

Aujourd'hui, c'est à vous que revient cet immense honneur que d'y
planter la bannière royale.

Soyez grands, soyez forts, soyez courageux et défendez nos valeurs
face à nos adversaires.\\

Et n'oubliez pas Chevalier~:

\begin{center}
  \textsc{L'extinction n'est pas une option.}
\end{center}

\subsection{Le champ de bataille}

Bien que d'origine 100\%{} naturelle\footnote{et élevée en plein air,
  label rouge}, la Terre du Centre se découpe admirablement bien en
cases carrées. Parce que notre bon Roy est également fin stratège, il
vous indiquera à chaque fois quelle zone vous devrez conquérir et
défendre.
\newpage

\section{Les Règles de l'Engagement}

\subsection{Le champ de bataille}
\prolocitation{--- Et ça, c'est la Terre du Centre~?\\
--- Effectivement. C'est la Terre du Centre.\\
--- Elle n'a pas l'air terrible.\\
--- Faut pas s'y fier, car personne n'en est ressorti~!}

Les territoires de la Terre du Centre sont composés de différentes
cases, pour différents terrains. Vous serez ainsi amené à rencontrer :

\begin{itemize}
  \item de l'\textbf{herbe}, terrain de base qui ne procure ni
    avantage ni handicap ;
  \item des \textbf{routes}~: le déplacement sur une route ajoute un
    bonus à la distance maximale de déplacement ;
  \item des \textbf{forêts}~: les forêts sont épaisses dans la Terre du
    Centre, et limitent fortement le champ de vision ;
  \item des \textbf{marais}~: le déplacement dans les marais est
    difficile, et impose un malus à la distance maximale de
    déplacement ;
  \item des \textbf{murs}~: dressés par des entités
    surnaturelles\footnote{selon la légende, elles porteraient des
      tuniques noires}, les murs sont absolument infranchissables (et
    opaques, bien entendu) ;
  \item des \textbf{tours de guet} : une tour de guet apporte des
    bonus en distance de vision et portée de tir.
\end{itemize}

\subsection{Vos hommes}
\prolocitation{}

Toutes les unités peuvent attaquer, au maximum une fois par tour.

\begin{itemize}

\item \textbf{Le voleur}

\prolocitation{Citation not found}

Le voleur

Points d'Action : //FIXME

Type d'attaque : corps-à-corps (1 case)

\item \textbf{Le barbare}

\prolocitation{--- Tiens donc, un paysan\\
--- Chuis un barbare~!\\
--- Aucune différence~!\\
(NDLR\footnote{Devant la violence de cette scène et la présence de
    mineurs parmi notre public, nous avons fait le choix de ne pas
    diffuser ces images. Merci de votre compréhension.})}

Le barbare se déplace à vitesse normale (FIXME nb Pt / case), et a un
champ de vision standard (FIXME cases, image).

Points d'Action : //FIXME

Type d'attaque : dégâts de zone

\item \textbf{Le tireur d'élite}

\prolocitation{Citation not found}

Le tireur d'élite est une unité lente

Points d'Actions : //FIXME

Type d'attaque :

\end{itemize}

\subsection{Le champ de vision}
\prolocitation{Loin des yeux, loin du c\oe{}ur\ldots{}}

\subsection{Les déplacements}
\prolocitation{--- Descendez l'escalier, et c'est tout droit.\\
--- Mais ça veut dire que\ldots{} Si on avait pris tout droit en
entrant, on serait arrivé ici directement ?
--- Bien sûr !}

\newpage
\section{Conclusion}

\prolocitation{End of Line, man.}
